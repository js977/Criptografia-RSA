\chapter{Conclusões}
\label{sec:conclusoes}


A cifra clássica representa um ponto de partida fundamental na história da criptografia, servindo como base para o desenvolvimento de métodos mais complexos e sofisticados de proteção de informações. Por meio da Cifra de César, cifra de deslocamento simples, um dos primeiros métodos conhecidos para codificar mensagens, demonstrou-se a importância de ocultar informações sensíveis para protegê-las de terceiros. 

No entanto, vimos que essas cifras são, nos dias de hoje, completamente inseguras  o que levou a criptografia a evoluir para algoritmos mais complexos tal como o RSA.


O método de RSA é hoje um dos métodos de base nos sistemas critográficos.

Como verificamos no secção~\ref{sec:criptoanalise}, os métodos aí estudados são incapazes de quebrar o método RSA para chaves com um comprimento superior a 30bits. As chaves seguras actuais para o método RSA estão actualmente na ordem do 2048 bits.\footnote{Ver página: \emph{Size considerations for public and private keys}, \url{https://www.ibm.com/docs/en/zos/2.3.0?topic=certificates-size-considerations-public-private-keys}}

Para a tentativa de quebrar a cifra RSA os métodos actuais são os métodos do crivo quadrático, assim como aproximações usandos computadores quânticos (algoritmo de Shor).

No semestre seguinte, dedicaremos uma análise mais aprofundada à criptoanálise do método RSA, com especial ênfase no crivo quadrático

