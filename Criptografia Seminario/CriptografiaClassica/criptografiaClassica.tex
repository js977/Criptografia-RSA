\chapter{Criptografia Clássica}

Designa-se usualmente por \emph{criptografia clássica} as cifras  pré-computacionais, desenvolvidas e utilizadas tendo por base processos mecânicos ou manuais. O mais simples deste tipo de criptografia consiste em trocar uma letra pela seguinte. Um código similar foi usado por Júlio César, cuja a chave era estabelecida pelo deslocamento, de três posições, nas letras do alfabeto~\cite{Coutinho2005}.

\begin{definicao}[Cifra Deslocamento]
 Seja $\mathcal{M}=\mathcal{C}=\mathbb{Z}_{26}^*$, $\mathcal{K}=\mathbb{Z}_{26}$. 7
 Para $0 \leq K \leq \mid\mathbb{Z}_{26}\mid$=26, define-se:
$$e_k(x)=(x+K) \; mod \; 26$$ e $$d_k(y)=(y - K) \; mod \; 26$$ para todo o $x,y \in \mathbb{Z}_{26}$
\end{definicao}

\emph{Nota:} para $\mathbb{Z}_{n}$ tem-se $n=26$ uma vez que o alfabeto adotado tem 26 caracteres.

\begin{teorema}[Cifra Deslocamento Simples] As funções $e_k$ e $d_k$ constituem uma cifra.
\end{teorema}
\begin{demonstracao}
    \begin{align*}
    d_k(e_k(x)&=d_k(x+k)=\\
              &=(x+k)-k=\\
              &=x\;mod\;26
\end{align*}

Então, por definição,a cifra de deslocamento é uma cifra.
\end{demonstracao}


\begin{definicao}[Cifra Deslocamento Linear]
    Seja $\mathcal{M}=\mathcal{C}=\mathbb{Z}_{26}^*$, e seja:
$K=\{(a,b) \in \mathbb{Z}\times \mathbb{Z} : mdc(a,26)=1\}$.

Para $K=(a,b) \in \mathcal{K}$, define-se:
$$ e_k(x)=(ax + b)\;mod\;26$$
e
$$ d_k(y)=a^{-1}(y - b)\;mod\;26$$
para todo o $x,y \in \mathbb{Z}_{26}$
\end{definicao}

\begin{teorema}[Cifra deslocamento linear] As funções $e_k$ e $d_k$ constituem uma cifra.
\end{teorema}
\begin{demonstracao}
    \begin{eqnarray*}
    d_k(e_k(x)) & = & d_k(ax +b)\\
                & = & a^{-1}(ax+b -b)\quad\text{\;por definição $a$ é invertível em $\mathbb{Z}_{26}$}\\
                & = & a^{-1}ax\\
                & = & x
    \end{eqnarray*} 
Por definição, verifica-se que o resultado apresentado anteriormente é verdadeiro.
\end{demonstracao}


Relativamente à criptoanálise, as cifras clássicas são cifras muito fracas ou completamente inseguras pelo que estão suscetíveis a ataques por procura exaustiva. Também é possível quebrar esta cifra por ataques baseada na frequência relativa das letras, diagramas,trigramas, letras iniciais e finais das palavras.

Por exemplo, se todas as todas as ocorrências da letra $a$ são substituídas pela letra $x$, uma mensagem cifrada contendo muitas instâncias da letra $x$, iria sugerir ao criptoanalista, que a letra $x$ representa a letra $a$.

De facto, para uma dada linguagem verifica-se que cada letra aparece de acordo com uma frequência própria. No caso da língua portuguesa tem-se a seguinte tabela de frequência:~\cite{Quaresma2009a}:

\begin{table}[h]
\centering
\begin{tabular}{>{\columncolor{gray!20}}cccccccccc}
a & 12,71\% & \cellcolor{gray!20}b & 0.81\% & \cellcolor{gray!20}c & 4,16\% & \cellcolor{gray!20}d  & 5,52\% & \cellcolor{gray!20}e & 11,99\% \\
f & 1,43\% & \cellcolor{gray!20}g & 1,32\% & \cellcolor{gray!20}h & 0,74\% & \cellcolor{gray!20}i  & 7,18\% & \cellcolor{gray!20}j & 0,21\% \\
k & 0,00\% & \cellcolor{gray!20}l & 3,23\% & \cellcolor{gray!20}m & 4,48\% & \cellcolor{gray!20}n  & 5,24\% & \cellcolor{gray!20}o & 11,32\% \\
p & 3,07\% & \cellcolor{gray!20}q & 1,41\% & \cellcolor{gray!20}r & 6,47\% & \cellcolor{gray!20}s  & 7,99\% & \cellcolor{gray!20}t & 5,31\% \\
u & 3,44\%  & \cellcolor{gray!20}v & 1,36\% & \cellcolor{gray!20}w & 0.02\% & \cellcolor{gray!20}x  & 0,28\% & \cellcolor{gray!20}y & 0,02\% \\
z & 0,37\% & & & & & &  & & \\
\end{tabular}
\end{table}

Com estes dados, agrupa-se estes valores em grupos, das letras com maior frequência para as menos frequentes:

\begin{table}[h]
\centering
\begin{tabular}{>{\columncolor{gray!20}}ccccc}
Primeiro grupo  & \cellcolor{gray!20}Segundo grupo & \cellcolor{gray!20}Terceiro grupo & \cellcolor{gray!20}Quarto grupo  & \cellcolor{gray!20}Quinto grupo\\
a,e,o &s,r,i & n,d,m,u,t,c &  l,p,v,g,h,q,b,f  & z,j,x,k,w,y\\
\end{tabular}
\end{table}

Notar que este processo reduz o número de tentativas e erro a realizar antes de se conseguir quebrar o código tornando-o mais eficiente.

Logo para quebrar a cifra utilizada para obter o texto encriptado ``a fkdyh whp gh vhu pdqwlgd', sabendo que foi utilizado um sistema de cifração de deslocamento simples, precisamos de calcular as frequências relativas para o texto cifrado,tem-se:


\begin{table}[h]
\centering
\begin{tabular}{>{\columncolor{gray!20}}cccccccccc}
d & 17,9\% & \cellcolor{gray!20}f & 7,1\% & \cellcolor{gray!20}g & 7,1\% & \cellcolor{gray!20}h & 21,4\% \\
k & 3,6\% & \cellcolor{gray!20}l & 3,6\% & \cellcolor{gray!20}p & 7,1\% & \cellcolor{gray!20}q & 3,6\% \\
u & 7,1\% & \cellcolor{gray!20}v & 7,1\% & \cellcolor{gray!20}w & 10,7\% & \cellcolor{gray!20}y & 3,6\% \\
\end{tabular}
\end{table}
Por análise das frequências relativas e consultado o primeiro grupo, onde as letras tem maior frequência relativa, uma vez que ``d'', com $17,9\%$, e o ``h'', com $21,4\%$ então podemos fazer ``d'' corresponder a ``a'', por exemplo. Neste caso, a nossa chave é 3.

Assim, obtemos o texto claro ``a chave tem de ser mantida secreta''. Logo a cifra foi quebrada.~\cite{Quaresma2009a}

Com recurso do computador podemos facilmente realizar todo o processo anterior rapidamente pelo que este tipo de cifra é muito fraca.