% ************************** Thesis Abstract *****************************
% Use `abstract' as an option in the document class to print only the titlepage and the abstract.
\begin{abstract}
\ifisLangEn


% ************************************************************************
%
% As linhas seguintes definem o `resumo' em portugu\^{e}s apenas no caso
% da l\'{\i}ngua utilizada ser o ingl\^{e}s. Caso contr\'{a}rio, ser\~{a}o ignoradas.
%
% ************************************************************************

  \cleardoublepage
  \setsinglecolumn
  \chapter*{\centering \Large Resumo}
  \thispagestyle{empty}

\else
A criptografia RSA é um sistema de segurança digital que utiliza um par de chaves - uma pública e uma privada - para proteger informações durante a transmissão. Fundamentado na complexidade matemática da fatorização de números de grande dimensão. O RSA é muito utilizado para garantir a confidencialidade e autenticidade de dados em transações online, \emph{e-mails} seguros e na proteção de informações sensíveis. Sua utilidade reside na capacidade de enviar mensagens seguras para qualquer pessoa, usando uma chave pública disponível, enquanto apenas o destinatário autorizado, com acesso à chave privada correspondente, pode decifrar e ler a mensagem. Este método de criptografia eficaz continua a desempenhar um papel essencial na segurança da era digital.

Neste estudo, inicialmente introduzimos os princípios matemáticos essenciais para a compreensão dos sistemas criptográficos. Em seguida, discutimos os sistemas criptográficos, destacando especialmente a criptografia RSA, abordando a cifra RSA assim como a sua criptoanálise (incluindo Métodos de Fermat, Divisões e Euclides). Concluímos com uma análise comparativa dos diversos métodos mencionados anteriormente.

\fi
\end{abstract}
