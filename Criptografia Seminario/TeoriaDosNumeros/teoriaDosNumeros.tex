%*******************************************************************************
%****************************** Second Chapter *********************************
%*******************************************************************************



\chapter{Fundamentos Matemáticos}
\label{sec:FundamentosMatematicos}

Em seguida são apresentados fundamentos teóricos matemáticos indispensáveis ao estudo da criptografia.

\section{Funções Unidirecionais}

\begin{definicao}[Função Unidirecional]
\label{def:funcaoUnidirecional}
Uma função $f$ de um conjunto $X$ para um conjunto $Y$ é dita uma função unidirecional («one-way function») se $f(x)$ é «fácil de calcular» para todo o $x \in X$, mas «essencialmente para todos» os elementos $y \in Im(f)$ é «computacionalmente difícil» achar um $x \in X$ tal que $f(x)=y.$
\end{definicao}

\begin{definicao}[Função Unidirecional com Escapatória]
\label{def:funcaoUnidirecionalEscapatoria}
Uma função unidirecional com escapatória é uma função unidirecional $f: X \longrightarrow Y$ com a propriedade de que dado algum tipo de informação adicional torna-se possível encontrar, para um dado $y \in Im(f)$ e um dado $x \in X$ tal que $f(x)=y$.
\end{definicao}



%PEDIR AO PROFESSOR PARA VERIFICAR ESSA DEFINIÇÃO

\section{Resultados da Teoria dos Números}

\begin{definicao}[Divisibilidade]: 
\label{def:divisibilidade}
Dados $a,b \in \mathbb{Z}$, com $a\neq0$, diz-se que, $a$ divide $b$, e escreve-se $a|b$, se existe $q \in \mathbb{Z}$ tal que $b=aq$.
\end{definicao}

Convenção: Quando se escreve $a|b$ está implícito que a $a \neq 0$.

Se $a|b$ também se diz que $a$ é um divisor de b, que b é um múltiplo de $a$ ou que $b$ é divisível por $a$.

Se $a$ não divide $b$, escreve-se $a \nmid b$.

Para quaisquer $a,b,c \in \mathbb{Z}$ tem-se:

\begin{enumerate}
    \item{ $a|0, 1|a \;e\;a|a$};
    \item{ $a|b 	\Leftrightarrow a|-b \Leftrightarrow -a|b$};
    \item{$a|b \land b|c \Leftrightarrow a|c$};
    \item Para quaisquer $x,y \in \mathbb{Z}, a|b \land a|c \Leftrightarrow a|bx + cy$;
    \item $a|1 \Leftrightarrow a=\pm b$;
    \item $a,b \in \mathbb{N} \land a|b \Leftrightarrow a \leq b$;
    \item Um inteiro não nulo tem um número finito de divisores.
\end{enumerate}


\begin{teorema}[Algoritmo da Divisão Inteira] 
\label{teo:AlgoritmoDivisaoInteira}
Dados $a,b \in \mathbb{Z}$, com $a \neq 0$, existem $q,r \in \mathbb{Z}$, únicos, tais que $$b=aq, \; com \; 0\leq r < |a|$$
$q$ e $r$ são, respetivamente, o quociente e o resto da divisão inteira de $b$ por $a$.
\end{teorema}

Observações:

\begin{enumerate}
    \item $a|b$ se e só se o resto da divisão inteira de $b$ por $a$ é zero.
    \item Em $C/C++$, os operadores ``$/$'' e ``\%'' dão-nos o quociente e o resto da divisão inteira (desde que o divisor e o dividendo sejam inteiros).
\end{enumerate}

\begin{definicao}[Congruência $\mod m$]
\label{def:congruenciamodm}
Para $m \in \mathbb{N}$ a relação de congruência módulo $m$ é a relação definida em $\mathbb{Z}$ por $$a \equiv b (mod\;m) \Leftrightarrow  m|a-b, \; a,b \in \mathbb{Z}$$
Se $a \equiv b (\mod m)$ diz-se que $a$ é congruente módulo $m$ com $b$.
\end{definicao} 

Observe-se que $a \equiv b (mod\;m)$ se e só se $a$ e $b$ têm o mesmo resto quando dividos por $m$.

Propriedades da congruência

\begin{enumerate}
    \item $a \equiv a\;(mod\;m)$ (Reflexividade)
    \item $a \equiv b\;(mod\;m)\Rightarrow b\equiv a\;(mod\; m)$
    \item $a \equiv b\;(mod\;m) \land a \equiv b\;(mod\; m) \Rightarrow a\;\equiv c(mod\; m)$
    \item $a \equiv b\;(mod\;m) \land c \equiv d(mod\;m) \Rightarrow a + c \equiv b+d \;(mod\;m)$
    \item $a \equiv b\;(mod\; m) \land c \equiv d(mod\; m) \Rightarrow ac \equiv bd\;(mod\; m)$
    \item $a \equiv b\;(mod\; m)\Rightarrow mdc(a.m)=mdc\;(b,m)$
    \item $ab \equiv ac\;(mod\; m)\Rightarrow b \equiv c\;(mod \frac{m}{mdc\;(a,m)})$
\end{enumerate} 
Das propriedades 1, 2 e 3 resulta que, para $m \in \mathbb{N}$, a relação de congruência módulo $m$ é uma relação de equivalência em $\mathbb{Z}$.

As classes de equivalência desta relação de equivalência chamam-se classes de congruência módulo m.

A classe de congruência módulo $m$ a que pertence $a\in \mathbb{Z}$ é representada por $[a]_m$ ou $\overline{a}$.

Uma vez que $a \in \mathbb{Z}$ é congruente módulo $m$ com o resto da divisão inteira por $m$, e os $m$ restos possíveis são $0,1,2,\dotsc,m-2$ e $m-1$ classes, conclui-se que há $m$ classes de congruência módulo $m$:
$[0]_m,[1]_m,\dotsc,[m-1]_m$

Para $m \in \mathbb{N}$, por $\mathbf{Z}_m$ representa-se o conjunto das classes de congruência módulo $m$, isto é, $$\mathbb{Z}_m={[0]_m,[1]_m,\dotsc,[m-1]_m}$$
Uma vez que $$a \equiv c(mod\; m) \land b \equiv d(mod\; m) \Rightarrow a + b \equiv c + d(mod\; m)$$
pode definir-se uma operação em $\mathbb{Z}_m$(adição de classes de congruência) por $[a]_m + [a]_m =[a +b]_m$
$$(\mathbb{Z}_m,+) \; é \; um \; grupo \;abeliano$$
Neste grupo o elemento neutro é $[0]_m$ e o simétrico de $[a]_m$ é $[-a]_m=[m-a]_m$\\

\begin{definicao}[Máximo Divisor Comum] Sejam $a,b \in \mathbb{Z}$ com $a \neq 0$ ou $b \neq 0$ e considere-se $D=\{c \in \mathbb{Z}: c|a \land c|b\}$.
\end{definicao}

$D \neq \emptyset$, porque $1 \in D$ e $D$ é finito porque um inteiro não nulo tem um número finito de divisores.

Então $D$ tem um máximo ao qual se chama máximo divisor comum de $a$ e $b$. Esse máximo é representado por $mdc(a,b)$.

Se $mdc(a,b)=1$ diz-se que os inteiros $a$ e $b$ são primos entre si ou que $a$ é primo com $b$.

Propriedades dos máximos divisores comum

Para quaisquer $a,b,c \in \mathbb{Z} \backslash \{0\}$, tem-se:
\begin{enumerate}
    \item $mdc(a,b)=mdc(b,a)=mdc(a,-b);$
    \item $mdc \left(\frac{a}{mdc(a,b)},\frac{b}{mdc(a,b)}\right)=1$
    \item $mdc(a,b)$ é o menor elemento positivo de $\{ax + by:\;x,y \in \mathbb{Z}\}$
    \item Se $x,y \in \mathbb{Z}$ são tais que $mdc(a.b)=ax + by$ então $mdc(x,y)=1$;
    \item $mdc(a,b)$ é o único divisor comum, positivo, de $a$ e $b$ tal que:
    $$x \in \mathbb{Z}\backslash \{0\} \land x|a \land x|b \Rightarrow x|mdc(a.b)$$
    \item $a|bc \land mdc(a,b)=1 \Rightarrow a|c$
\end{enumerate}
Para calcular o máximo divisor comum de $a,b \in \mathbb{Z}\backslash\{0\}$ usa-se o algoritmo de Euclides.

\begin{teorema}[Algoritmo de Euclides] 
\label{teo:AlgoritmoEuclides}
Sejam $a \in \mathbb{N}$ e $b \in \mathbb{Z}$. Aplicando sucessivamente o algoritmo da divisão obtém-se: 

\begin{align*}
    & b=aq_1 + r_1,\;0 < r_1 < a\\
    & a=r_1q_2 + r_2,\;0 < r_2 < r_1\\
    &r_1=r_2q_3 + r_3, \;0 < r_3 < r_2\\
                \vdots\\
    &r_{k-2}=r_{k-1}q_k + r_k, \;0 < r_k < r_{k-1}\\
    &r_{k-1}=r_kq_{k+1}
\end{align*}
para um dado $k \in \mathbb{N}\;(r_0:=a\;e\;r_{-1}:=b)$

Então $mdc(a,b)=r_k$.
\end{teorema}

\begin{proposicao}[Congruências Lineares]
Sejam $m \in \mathbb{N},\; a,b \in \mathbb{Z}$ com $mdc(a,m)=1$. A congruência $ax \equiv b(mod\;m)$ tem solução e o conjunto das soluções é uma classe de congruência módulo $m$.
\end{proposicao}

Método de resolução de $ax \equiv b(mod\;m)$ com $mdc(a,m)=1$:

Usando o algoritmo de Euclides determinam-se $x_0,y_0 \in \mathbb{Z}$ tais que $ax_0+my_0=1$. De $ax_0\equiv1(mod\;m)$ resulta que $ax_0\equiv b(mod\;m)$.

O conjunto das soluções de $ax\equiv b(mod\;m)$ é $[x_0b]_m$.

\begin{proposicao}[Congruências Lineares---Caso geral] Sejam $m \in \mathbb{N},a,b\in\mathbb{Z}$ e $d=mdc(a,m)$. A congruência $ax\equiv b(mod m)$ tem solução se e só se $d|b$.
\end{proposicao}

Se $d|b$ então $$ax\equiv b(mod \; m) \Leftrightarrow \frac{a}{d}x\equiv\frac{b}{d}\left(mod\frac{m}{d}\right)$$
e o conjunto das soluções é a união de d classes de congruência módulo m.\\

\begin{proposicao}[Inverso Multiplicativo]: Se $mdc(a,m)=1$, então $[a]_m$ é invertível em $\mathbb{Z}_m$ e , sendo $x_0,y_0 \in \mathbb{Z}$ tais que $ax_0+my_0=1$, tem-se que:$$[a]_m^{-1}=[x_0]_m$$
\end{proposicao}

Notar que se $mdc(a,m) > 1, [a]_m$ não é invertível\\
Os elementos invertíveis em $(\mathbb{Z}_m,.)$ são os elementos de $\{[a]_m:mdc(a,m)=1\}$.\\
Observação: $(\mathbb{Z}_m\backslash \{[0]_m\},.)$ é um grupo, se e só se $m$ é primo.\\

\begin{teorema}[Grupo multiplicativo]Seja $U_m=\{[a]_m:mdc(a,m)=1\}$. Consideremos $m \in \mathbb{N}$. $U_m$ é um grupo para a multiplicação de classes de congruência módulo $m$.
\end{teorema}

Notações:
\begin{enumerate}
    \item Quando se trabalha em $\mathbb{Z}_m$ muitas vezes representa-se $[a]_m$ apenas por $r$, sendo \mbox{$r \in \{0,1,\ldots,m-1\}$} o resto da divisão inteira por $a$ por $m$;
    \item Sendo $a \in \mathbb{Z}$ e $m \in \mathbb{N}$, é usual representar por $amod\;m$ o resto da divisão inteira de $a$ por $m$;
    \item Se $mdc(a,m)=1$, por $a^{-1} \mod\;m$ representa-se o inverso de $[a]_m$ em $\mathbb{Z}_m$
\end{enumerate}

\begin{definicao}[Menor Múltiplo Comum] Dados $a,b \in \mathbb{Z}\backslash \{0\}$, um inteiro não nulo $c$ é um múltiplo comum de $a$ e $b$ se $a|c$ e $b|c$.
\end{definicao}

Sejam $a,b \in \mathbb{Z}\backslash \{0\}$ e considere-se $M=\{c \in \mathbb{N}:a|c \land b|c\}$.\\

$M \neq \emptyset$ porque $|ab| \in M$. Além disso, $M \subseteq \mathbb{N}$. Então $M$ tem um mínimo ao qual se chama menor múltiplo comum de $a$ e $b$.\\
Esse mínimo é representado por $mmc(a,b).$\\

Para quaisquer $a,b \in \mathbb{Z} \backslash \{0\}$, tem-se:
\begin{enumerate}
    \item $mmc(a,b)$ é o único múltiplo comum, positivo, de $a$ e $b$ tal que:$$x \in \mathbb{Z} \land a|x \land b|x \Rightarrow mmc(a,b)|x$$
    \item Se $n \in \mathbb{N}$ é um divisor comum de $a$ e $b$ então
    $$mmc\left(\frac{a}{n},\frac{b}{n}\right)$$
    \item $mmc(a,b)mdc(a,b)=|ab|$
\end{enumerate}
O menor múltiplo comum de $a,b \in \mathbb{Z} \backslash \{0\}$ pode ser calculado usando o algoritmo de Euclides e a propriedade 3.\\

\begin{definicao}[máximo divisor comum---caso geral] 
$n \in \mathbb{N},n \ge 2,a_1,a_2,\dotsc,a_n \in \mathbb{Z}$ não todos nulos. O máximo divisor comum de $a_1,a_2,\dotsc,a_n$ é o menor dos divisores comuns positivos de $a_1,a_2,...,a_n$. Representa-se por $mdc(a_1,a_2,\dotsc,a_n)$


Propriedades:
\begin{enumerate}
    \item $mdc(a_1,a_2,\dotsc,a_n)$ é o menor inteiro positivo da forma\\$a_1x_1+a_2x_2+\dotsc+a_nx_n$, com $x_1,x_2,\dotsc,x_n \in \mathbb{Z}$
    \item $mdc(a_1,a_2,\dotsc,a_n)$ é o único divisor comum positivo, de $a_1,a_2,\dotsc,a_n$ que é múltiplo de qualquer divisor comum de $a_1,a_2,\dotsc,a_n$
    \item $mdc(a_1,a_2,\dotsc,a_n)=mdc(mdc(a_1,a_2,\dotsc,a_{n-1}),a_n)$
\end{enumerate}
\end{definicao}

\begin{definicao}
Os inteiros $a_1,a_2,\dotsc,a_n$ são primos dois a dois se $mdc(a_i,a_j)=1$, para $i,j=1,2,\dotsc,n$, com $i \neq j$.
$$a_1,a_2,\dotsc,a_n \text{ são primos dois a dois}$$
$$\downarrow $$
$$a_1,a_2,\dotsc,a_n \text{ são primos entre si}$$
Notar que a implicação reciproca é falsa.
\end{definicao}

\begin{definicao}: $n \in \mathbb{N},n \ge 2,a_1,a_2,\dotsc,a_n \in \mathbb{Z}$ não todos nulos. O menor múltiplo comum de $a_1,a_2,\dotsc,a_n$ é o menor dos múltiplos comuns positivos de $a_1,a_2,\dotsc,a_n$. Representa-se por \linebreak $mmc(a_1,a_2,\dotsc,a_n)$.\\

Propriedades:
\begin{enumerate}
    \item $mmc(a_1,a_2,\dotsc,a_n)$ é o único múltiplo comum, positivo, de $a_1,a_2,\dotsc,a_n$ que divide qualquer múltiplo comum de $a_1,a_2,\dotsc,a_n$;
    \item $mmc(a_1,a_2,\dotsc,a_n)=mmc(mmc(a_1,a_2,\dotsc,a_{n-1}),a_n)$
\end{enumerate}
Para $n \leq 3$, em geral,\\$mmc(a_1,a_2,\dotsc,a_n)=mmc(mmc((a_1,a_2,\dotsc,a_{n-1}),a_n) \neq |a_1,a_2,\dotsc,a_n|$
\end{definicao}

\begin{definicao} Um inteiro $p>1$ diz-se um número primo se os únicos divisores positivos de $p$ são 1 e $p$.
Um inteiro diz-se composto se não é primo.
\end{definicao}

\begin{teorema} $p$ número primo; $a_1,a_2,\dotsc,a_n \in \mathbb{Z}$ $$p|a_1a_2\dots ca_n\;\Rightarrow \;p|a_1 \lor p|a_2\lor \dots c \lor p|a_n$$
\end{teorema}

\begin{teorema}[Teorema Fundamental da Aritmética] Todo o inteiro maior que $1$ pode ser escrito, de modo único(a menos da ordem dos fatores), como produto de números primos.
\end{teorema}

\begin{teorema}[Fatorização de Números Primos] Se $a=p_1^{\alpha_1}p_2^{\alpha_2}\dotsc p_k^{\alpha_k}$ e $a=p_1^{\beta_1}p_2^{\beta_2}\dotsc p_k^{\beta_k}$ onde $p_1,\dotsc p_k$ são números primos dois a dois e $\alpha_1,\dotsc,\alpha_k$,$\beta_1,\dotsc,\beta_k \in \mathbb{N}_0$, então  $$a|b \Leftrightarrow(\alpha_i \leq \beta_i,i=1,\dotsc,k)$$
$$mdc(a,b)=p_1^{min\{\alpha_1,\beta1\}}p_2^{min\{\alpha_2,\beta2\}}\dotsc p_k^{min\{\alpha_k,\beta_k\}}$$
e
$$mdc(a,b)=p_1^{max\{\alpha_1,\beta1\}}p_2^{max\{\alpha_2,\beta2\}}\dotsc p_k^{max\{\alpha_k,\beta_k\}}$$
\end{teorema}

\begin{definicao}[Função de Euler]: a função de Euler é a função $\phi : \mathbb{N} \Rightarrow \mathbb{N}$ definida por:$$\phi ( n)=|{a \in \mathbb{N}:a \leq n\;e \;mdc(a,n)=1}|,\;n\in \mathbb{N}$$
\end{definicao}

\begin{teorema}
\label{teo:teoremaA}
 A função $\phi$ é multiplicativa, isto é, se $m,n \in \mathbb{N}$ são tais que $mdc(m,n)=1$, então $$\phi(mn)=\phi(m) \phi(n)$$
 \end{teorema}
 
\begin{teorema}    
 Sejam, $p_1,p_2,\dotsc ,p_n$ números primos distintos dois a dois e\\ $\alpha_1,\alpha_2,\dotsc,\alpha_n \in \mathbb{N}$
$$\phi(p_1^{\alpha_1} \dotsc p_k^{\alpha_k} )=(p_1^{\alpha_1} -p_1^{\alpha-1} )\dotsc (p_k^{\alpha_k} -p_k^{\alpha-1} )$$
\end{teorema}
\begin{teorema}[Pequeno Teorema de Fermat]
\label{teo:PequenoTeoremaFermat}
Se $n$ é um número primo, então $a^{n-1} \equiv 1(mod n)$, para todo o $a \in \mathbb{Z}$ tal que $mdc(a,n)=1$
\end{teorema}
\begin{teorema}[Teorema Chinês dos Restos]
\label{teo:TeoremaChinesDosRestos}
Sejam $m_1,m_2,\dotsc,m_k \in \mathbb{Z}$ primos dois a dois e $a_1,a_2,\dotsc,a_k \in \mathbb{Z}$.

O sistema 
\begin{equation}
\left\{ \begin{aligned} 
  x &\equiv a_1 (mod \;m_1)\\
  \vdots\\
  x &\equiv a_K (mod\; m_k)
\end{aligned} \right.
\end{equation}
tem solução.
\end{teorema}
Seja $m=m_1m_2\dotsc \; m_k.$ Para $i=1,2,\dotsc,m_k$ seja $b_i \in \mathbb{Z}$ tal que $\frac{m}{m_i}b_i\equiv 1(mod\; m_i)$ e considere-se 
$$x_0=\sum_{i=1}^k \frac{m}{m_i}a_ib_i$$
O conjunto das soluções é $[x_0]_m$.

