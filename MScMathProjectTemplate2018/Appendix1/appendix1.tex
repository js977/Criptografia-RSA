% ******************************* Thesis Appendix A ********************************
\chapter{More information $\int$}

\section*{Carl Friedrich Gauss}

Johann Carl Friedrich Gauss (30 April 1777 -- 23 February 1855) was a German mathematician who contributed significantly to many fields, including number theory, algebra, statistics, analysis, differential geometry, geodesy, geophysics, electrostatics, astronomy, matrix theory, and optics.

Sometimes referred to as the Princeps mathematicorum (Latin, ``the Prince of Mathematicians'' or ``the foremost of mathematicians'') and ``greatest mathematician since antiquity'', Gauss had a remarkable influence in many fields of mathematics and science and is ranked as one of history's most influential mathematicians.

Gauss was a child prodigy. There are many anecdotes about his precocity while a toddler, and he made his first ground-breaking mathematical discoveries while still a teenager. He completed Disquisitiones Arithmeticae, his magnum opus, in 1798 at the age of 21, though it was not published until 1801. This work was fundamental in consolidating number theory as a discipline and has shaped the field to the present day.

Gauss's intellectual abilities attracted the attention of the Duke of Brunswick, who sent him to the Collegium Carolinum (now Braunschweig University of Technology), which he attended from 1792 to 1795, and to the University of G\"{o}ttingen from 1795 to 1798. While at university, Gauss independently rediscovered several important theorems; his breakthrough occurred in 1796 when he showed that any regular polygon with a number of sides which is a Fermat prime (and, consequently, those polygons with any number of sides which is the product of distinct Fermat primes and a power of 2) can be constructed by compass and straightedge. This was a major discovery in an important field of mathematics; construction problems had occupied mathematicians since the days of the Ancient Greeks, and the discovery ultimately led Gauss to choose mathematics instead of philology as a career. Gauss was so pleased by this result that he requested that a regular heptadecagon be inscribed on his tombstone. The stonemason declined, stating that the difficult construction would essentially look like a circle.

The year 1796 was most productive for both Gauss and number theory. He discovered a construction of the heptadecagon on 30 March. He further advanced modular arithmetic, greatly simplifying manipulations in number theory. On 8 April he became the first to prove the quadratic reciprocity law. This remarkably general law allows mathematicians to determine the solvability of any quadratic equation in modular arithmetic. The prime number theorem, conjectured on 31 May, gives a good understanding of how the prime numbers are distributed among the integers.

\section*{Another section}

\subsection*{Subsection}

\subsubsection*{Subsubsection}
...
