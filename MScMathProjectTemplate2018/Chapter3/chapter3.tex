%*******************************************************************************
%*********************************** Third Chapter *****************************
%*******************************************************************************


\chapter{My third chapter}  %Title of the Third Chapter

\ifpdf
    \graphicspath{{Chapter3/Figs/Raster/}{Chapter3/Figs/PDF/}{Chapter3/Figs/}}
\else
    \graphicspath{{Chapter3/Figs/Vector/}{Chapter3/Figs/}}
\fi


%********************************** %First Section  **************************************
\section{Title with math \texorpdfstring{$\sigma$}{[sigma]}} %Section - 1.1

The well known Pythagorean theorem \(x^2 + y^2 = z^2\) was
proved to be invalid for other exponents.
Meaning the next equation has no integer solutions:
$x^n + y^n = z^n.$

The binomial coefficient is defined by the next expression:
\[
    \binom{n}{k} = \frac{n!}{k!(n-k)!}
\]

And of course this command can be included in the normal
text flow \(\binom{n}{k}\). Limit $\lim_{x\to\infty} f(x)$ inside text.
$$\lim_{x\to\infty} f(x)$$

The most famous equation in the world: $E^2 = (m_0c^2)^2 + (pc)^2$, which is
known as the \textbf{energy-mass-momentum} relation as an in-line equation.

\begin{align}
CIF: \hspace*{5mm}F_0^j(a) = \frac{1}{2\pi \iota} \oint_{\gamma} \frac{F_0^j(z)}{z - a} dz
\end{align}

Integral $\int_{a}^{b} x^2 dx$ inside text.

\begin{align}
\iiint_V \mu(u,v,w) \,du\,dv\,dw
\end{align}


%********************************** %Second Section  *************************************
\section{Preliminaries I. Free constructions} %Section - 1.2

We will work with point-free real numbers as they are usually described in literature, that is, by generators subject to relations. Since the free generators come from a set that is in fact a meet-semilattice (while its elements are used in the free construction simply as elements of a set) we think that it may be useful for the reader to compare the free frames over sets with free frames over semilattices.

We will work with point-free real numbers as they are usually described in literature, that is, by generators subject to relations. Since the free generators come from a set that is in fact a meet-semilattice (while its elements are used in the free construction simply as elements of a set) we think that it may be useful for the reader to compare the free frames over sets with free frames over semilattices.



\subsection{Free semilattice with 1.} For a set $X$ define
$
F(X)=\{A\subseteq X\mid A\ \text{finite}\}
$
ordered by $\leq\;=\;\supseteq$ so that we have the meet $A\wedge B=A\cup B$. Denote by $\beta_X$ the mapping
$$
\beta_X=(x\mapsto\{x\})\colon X\to F(X).
$$
 Then we have for each meet-semilattice $S$ with 1 and each mapping $f\colon X\to S$ precisely one meet-semilattice homomorphism $\overline{f}\colon F(X)\to S$ such that $\overline{f}\beta_X=f$ and $\overline{f}(\emptyset)=1$, namely the homomorphism defined by $\overline{f}(A)=\bigwedge_{x\in A} f(x)$.

\subsection{Free frame generated by a semilattice with 1.} For a meet-semilattice  $S$ with 1  set
$
{\mathfrak D}(S)=\{U\subseteq S\mid\downarrow\! U=U\neq \emptyset\}.
$
${\mathfrak D}(S)$ is a frame with unions for joins and intersections for meets and if we denote by $\alpha_S$ the mapping
$$
\alpha_S=(s\mapsto\downarrow\! s)\colon S\to {\mathfrak D}(S)
$$
we have a meet-semilattice homomorphism such that for each frame $L$ and each meet-semilattice homomorphism $h\colon S\to L$ there is precisely one frame  homomorphism $\widehat h\colon {\mathfrak D}(S)\to L$ such that $\widehat h\alpha_S=h$, namely that defined by $\widehat h(U)=\bigvee_{s\in U}h(s)$.

The free frame over a set can be now obtained combining $F$ and $\mathfrak D$, that is, as $\mathfrak{D}F(X)$.

\subsection{Free frames over a set and over a meet-semilattice compared.}
Well, one may go on adding more and more subsections, but these are enough to illustrate how this works!


%********************************** % Third Section  *************************************
\section{Where does it come from?}  %Section - 1.3
\label{section1.3}

And this illustrates how one may add more sections to the text.


 \section{Next section}

 \section{Next section}


