% ************************** Thesis Abstract *****************************
% Use `abstract' as an option in the document class to print only the titlepage and the abstract.
\begin{abstract}
\ifisLangEn

If the Dissertation is in Portuguese, do not write here anything.
This is where you write your abstract in English.

% ************************************************************************
%
% As linhas seguintes definem o `resumo' em portugu\^{e}s apenas no caso
% da l\'{\i}ngua utilizada ser o ingl\^{e}s. Caso contr\'{a}rio, ser\~{a}o ignoradas.
%
% ************************************************************************

  \cleardoublepage
  \setsinglecolumn
  \chapter*{\centering \Large Resumo}
  \thispagestyle{empty}

\else
  %Aqui escreve o Resumo em Portugu\^{e}s no caso da disserta\c{c}\~{a}o usar o Ingl\^{e}s.

    A criptografia RSA revolucionou a segurança digital com o seu procedimento de chaves pública e privada, em contraponto com as cifras de chaves simétricas cuja gestão das chaves é muito difícil de gerir em ambientes com muitos intervenientes. A criptoanálise, por sua vez, desempenha um papel fundamental na avaliação e no aprimoramento da segurança dos sistemas criptográficos, buscando constantemente formas de tornar as comunicações digitais mais seguras e protegidas contra ameaças cibernéticas.

    Neste texto apresentam-se as implementações dos diferentes métodos descritos no trabalho de seminário, nomeadamente, os algoritmos para os métodos de deslocamento simples, linear e RSA.
    
    No final, foi feito um estudo comparativo dos métodos acima mencionados.
\fi
\end{abstract}
