\chapter{Conclusões}
\label{sec:Conclusoes}
Ao longo do trabalho constatou-se o procedimento eficaz e seguro para criptografar a informação desempenhada  pela sistema RSA.

De facto, verifica-se na secção~\ref{sec:CriptoanaliseRSA} que os métodos analisados não têm capacidade de ``quebrar'' o método RSA em chaves maiores que 20 dígitos. As chaves seguras utilizadas atualmente para o algoritmo RSA têm cerca de 2048 bits.\footnote{Ver página: \emph{Size considerations for public and private keys}, \url{https://www.ibm.com/docs/en/zos/2.3.0?topic=certificates-size-considerations-public-private-keys}}

No próximo semestre, iremos aprofundar o estudo da criptoanálise do método RSA, com foco especial no crivo quadrático."



%%% Local Variables:
%%% mode: latex
%%% TeX-master: "../thesis"
%%% End:
